\documentclass{article}
\usepackage[utf8]{inputenc}
\usepackage[T1]{fontenc}
\usepackage{amsmath}
\usepackage{graphicx}
\graphicspath{ {./images/} }

\title{Inovativni elektromagnetski sustavi - projekt}
\author{Darko Janeković}

\begin{document}
\maketitle

\section{Implementacija}
Rješenje zadatka implementirano je u programskom jeziku Python 3 koristeći
vanjske biblioteke PETSc i SLEPc. Programska biblioteka PETSc koristi se za
općenite podatkovne strukture potrebne za efikasno rješavanje problema. S druge
strane SLEPc se koristi za rješavanje svojstvenog problema.

Skripta za rješavanje problema traži postavljanje dva parametra:
\begin{itemize}
    \item \texttt{dim} - Dimenzije valovoda u centimetrima. Primjerice za
modeliranje WR229 valovoda potrebno je postaviti:
    \texttt{dim = np.array([5.817, 2.903])}.
    \item \texttt{h} - Korak diskretizacije u centrimetrima. Korak
diskretizacije označava najmanji korak mreže u okviru metode konačnih razlika.
\end{itemize}

Program će na početku svog rada stvoriti podatkovne strukture. Na prvom
mjestu to je objekt \texttt{DMDA} koji brine o ispravnoj prealokaciji rijetko
popunjene matrice te o ispravnoj distribuciji problema u slučaju
implementacije paralelnog izvođenja (zbog jednostavnosti, ovdje to nije
korišteno).

Nakon stvaranja i popunjavanja matrice, potrebno je stvoriti objekt
\texttt{EPS} koji rješava svojstveni problem. Nakon što su svojstvene
vrijednosti i vektori pronađeni program radi nekoliko stvari:
\begin{itemize}
    \item Ispisuje svojstvenu vrijednost i pripadajuću grešku.
    \item Sprema grafički prikaz razdiobe $z$ komponente električnog polja.
    \item Računa frekvenciju $f_0$ pri kojoj mod počinje propagirati.
    \item Sprema grafički prikaz $k_z(f)$ za sve izračunate svojstvene vrijednosti.
\end{itemize}

\section{Demonstracija pokretanja}

U ispisu u nastavku prikazano je pokretanje skripte. Grafički prikazi razdiobe
polja su spremljeni u trenutnom direktoriju.

\begin{verbatim}
$ python3 waveguide.py
=============================================================================
 Problem dimenzija: [5.817 2.903] [cm]
 Korak diskretizacije u obje dimenzije: 0.05
 Problem rezultira matricom dimenzija [116  58]
=============================================================================
 1. Svojstvena vrijednost 0.003783204609551006, greška 7.937783972117381e-09
 2. Svojstvena vrijednost 0.0060213601677123925, greška 8.009617064752424e-09
 3. Svojstvena vrijednost 0.009749763612308877, greška 4.700250058685318e-10
 f0 = 58694951.12414197 Hz
 f0 = 74048818.405483 Hz
 f0 = 94225374.73295589 Hz
\end{verbatim}

\section{Rezultati}

U nastavku su prikazani rezultati ranije prikazanog pokretanja skripte.

\begin{figure}[h]
    \includegraphics[width=7cm]{mod_0.pdf}
    \includegraphics[width=7cm]{mod_1.pdf}
    \includegraphics[width=7cm]{mod_2.pdf}
    \includegraphics[width=7cm]{kz.pdf}
\end{figure}

\end{document}
