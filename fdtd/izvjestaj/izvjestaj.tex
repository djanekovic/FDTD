\documentclass{article}
\usepackage[utf8]{inputenc}
\usepackage[T1]{fontenc}
\usepackage{amsmath}

\title{Inovativni elektromagnetski sustav - projekt}
\author{Darko Janeković}

\begin{document}
\maketitle

\section{Stabilnost BEC numeričke sheme}

Za početak, potrebno je dokazati stabilnost numeričke sheme
\begin{equation}
    u_j^{n+1} + \frac{a \mu}{2}(u_{j+1}^{n+1} - u_{j-1}^{n+1}) = u_j^n.
\end{equation}

Stabilnost numeričke sheme dokazuje se koristeći Von Neumannovu metodu, a ona
se temelji na provjeravanju amplitude harmonika na izlazu iz numeričke sheme.
Uvrštavanjem k-tog harmonika

\begin{equation}
    u_j^n = \alpha_k \lambda^n e^{-i k j \Delta x}
\end{equation}

u numeričku shemu dobiva se:

\begin{gather}
    \alpha_k \lambda^{n+1} e^{-i k j \Delta x} + \frac{a \mu}{2}(
        \alpha_k \lambda^{n+1} e^{-i k (j + 1) \Delta x} -
        \alpha_k \lambda^{n+1} e^{-i k (j - 1) \Delta x}
    ) = \alpha_k \lambda^n e^{-i k j \Delta x} \nonumber \\
    \lambda_k + \frac{a \mu}{2}(
        \lambda_k e^{i k \Delta x} -
        \lambda_k e^{-i k  \Delta x}) = 1 \nonumber \\
    \lambda_k = \frac{1}{1 + i a \mu \cdot \sin(k \Delta x)} \nonumber \\
    \lambda_k = \frac{1 - i a \mu \cdot \sin(k \Delta x)}
        {1 + a^2 \mu^2 \cdot \sin^2(k \Delta x)}
\end{gather}

Shema je stabilna ukoliko vrijedi $|\lambda_k|^2 \le 1$.

\begin{gather}
    |\lambda_k|^2 = \frac{1}{(1 + a^2 \mu^2 \cdot \sin^2(k \Delta x))^2} +
        \frac{a^2 \mu^2 \cdot \sin^2(k \Delta x)}
            {(1 + a^2 \mu^2 \cdot \sin^2(k \Delta x))^2} \\
    |\lambda_k|^2 = \frac{1}{1 + a^2 \mu^2 \cdot \sin^2(k \Delta x)}
\end{gather}

Lako je uočljivo da je shema bezuvjetno stabilna budući da za svaki $a$ odnosno
$\mu$, $|\lambda_k|^2$ ne postaje veći od $1$.

\section{Numerička disipacija}

Savršena numerička shema imat će $|\lambda_k| \approx 1$ jer u tom slučaju neće
doći do numeričke disipacije. Drugim riječima, što je $|\lambda_k|$ iznosom
manji, to će numerička shema više prigušivati numeričko rješenje.

Izvod reda veličine numeričke disipacije nastavlja se na izvod stabilnosti
numeričke sheme.

Potrebno je naći ovisnost faktora pojačanja u ovisnosti o $\phi_k$ gdje je
$\phi_k$ fazna brzina k-tog harmonika.

\begin{align*}
    |\lambda_k|^2 &= \frac{1}{1 + a^2 \mu^2 \cdot \sin^2(\phi_k)} \\
    &\approx 1 - a^2 \mu ^2 \cdot \sin^2(\phi_k) +
        \left(a^2 \mu ^2 \cdot \sin^2(\phi_k)\right)^2 -
        \left(a^2 \mu ^2 \cdot \sin^2(\phi_k)\right)^3... \\
    &\approx 1 -
        a^2 \mu^2 \cdot
        \left(\phi_k^2 - \frac{1}{3} \phi_k^4 + \frac{2}{45} \phi_k^6\right) +
        a^4 \mu^4 \cdot
        \left(\phi_k^2 - \frac{1}{3} \phi_k^4 + \frac{2}{45} \phi_k^6\right)^2
        - ... \\
    &\approx 1 -
        \left(a^2 \mu^2 \phi_k^2 - \frac{a^2 \mu^2}{3} \phi_k^4\right)
        \left(1 - a^2 \mu^2 \phi_k^2 - \frac{a^2 \mu^2}{3} \phi_k^4\right)
        - ... \\
    &\approx 1 - a^2\mu^2 \phi_k^2 -
        \left(a^4 \mu^4 + \frac{a^2 \mu^2}{3}\right) \phi_k^4
        + \frac{2}{3} a^4 \mu^4 \phi_k^6 - \frac{1}{9}a^4 \mu^4 \phi_k^8
        - ... \\
    &= O\left(\phi_k^2\right)
\end{align*}

Budući da $|\lambda_k|^2$ ovisi o $\phi_k$ kvadratno, može se pisati da je
numerička disipacija reda $\phi_k$ odnosno

\begin{equation}
    \epsilon(k) = O(\phi_k).
\end{equation}

\section{Numerička disperzija}

S druge strane, disperzija označava pogrešku faze numeričke sheme i kao takva
proizlazi iz argumenta kompleksnog broja $\lambda_k$.

\begin{align*}
    arg(\lambda_k) &= -\arctan{(a \mu \sin(\phi_k))} \\
    &\approx -a \mu \sin(\phi_k)
        + \frac{1}{3} a^3 \mu^3 \sin^3(\phi_k)
        - \frac{1}{5} a^5 \mu^5 \sin^5(\phi_k) + ... \\
    &\approx -a \mu \cdot
        \left( \phi_k - \frac{1}{3!}\phi_k^3 + \frac{1}{5!}\phi_k^5 -...\right)
        + ...
\end{align*}

Relativna greška faze se onda može izraziti kao
\begin{align*}
    \frac{arg(\lambda_k) - (- \phi_k a \mu)}{-\phi_k a \mu} =
        \frac{(-\phi_k + \frac{1}{3!}\phi_k^3 - ...) + \phi_k}{-\phi_k} =
        \frac{(-1 + \frac{1}{3!}\phi_k^2 - ...) + 1}{-1} \approx
        -\frac{1}{3!}\phi_k^2
\end{align*}

Odnosno pokazali smo da je relativna fazna greška reda $O(\phi_k^2)$.

\section{Implementacija}

Programsko rješenje problema izvedeno je u programskom jeziku Python. Kao
vanjski modul je korištena programska biblioteka PETSc kroz modul petsc4py.
Modul je korišten za rješavanje i spremanje sustava nepoznanica.

\end{document}
