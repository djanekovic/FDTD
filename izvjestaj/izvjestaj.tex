\documentclass{article}
\usepackage{amsmath}

\title{Inovativni elektromagnetski sustav - projekt}
\author{Darko Janeković}

\begin{document}
\maketitle

\section{Stabilnost BEC numeričke sheme}

Za početak, potrebno je dokazati stabilnost numeričke sheme. Stabilnost
numeričke sheme dokazuje se koristeći Von Neumannovu metodu.

\begin{equation}
    u_j^{n+1} + \frac{a \mu}{2}(u_{j+1}^{n+1} - u_{j-1}^{n+1}) = u_j^n
\end{equation}

Analiza stabilnosti koristeći Von Neumannovu metodu se temelji na provjeravanju
amplitude harmonika na izlazu iz numeričke sheme. Uvrštavanjem k-tog harmonika u
numeričku shemu dobiva se:

\begin{gather}
    u_j^n = \alpha_k \lambda^n e^{-i k j \Delta x} \nonumber \\
    \alpha_k \lambda^{n+1} e^{-i k j \Delta x} + \frac{a \mu}{2}(
        \alpha_k \lambda^{n+1} e^{-i k (j + 1) \Delta x} -
        \alpha_k \lambda^{n+1} e^{-i k (j - 1) \Delta x}
    ) = \alpha_k \lambda^n e^{-i k j \Delta x} \nonumber \\
    \lambda_k + \frac{a \mu}{2}(
        \lambda_k e^{i k \Delta x} -
        \lambda_k e^{-i k  \Delta x}) = 1 \nonumber \\
    \lambda_k = \frac{1}{1 + a \mu cos(k \Delta x)}
\end{gather}

Shema je stabilna ukoliko vrijedi $\lambda_k \le 1$. Lako je uočljivo da je shema
bezuvjetno stabilna budući da za svaki $a$ odnosno $\mu$, $\lambda_k$ ne postaje
veći od $1$.

\end{document}
